\section{Face Recognition: A Literature Survey}

\entrydate{November 11, 2024} \\
Pattern Recognition, Face Detection, Face Recognition, Person Identification, Algorithms


\subsection{Overview}
Discusses overall progress in face recognition research field upto 2000s. It takes into account face recognition from different sources, like 2d, 3d, still images, moving images. It also goes into deep about different algorithm and approaches. Eigenface and Fisherface are mostly common mentioned type of faces. Limitations like pose variation, illumination problem, issues derived from unconstrained enviorment are discussed\cite{zhaoFaceRecognitionLiterature2003}.  

\subsection{Highlights}
\begin{itemize}
    \item Discusses various strong and weak points of approaches.
    \item Mentions different approaches and their respective efficiencies.
    \item Written concisely covering a wide range of approaches.
\end{itemize}

\subsection{Limitations}
\begin{itemize}
    \item Illumination problem, pose variation problem section is obscured.
    \item Covered approaches have limited amount of database to work with
    \item Eigen vector, related mathematical notations needed more explanation.
\end{itemize}

\subsection{Evaluation}
Mostly covers everything, gives critical reasoning. Different approaches are discussed thoroughly. Proper presentations of face recognition approaches is visible.


